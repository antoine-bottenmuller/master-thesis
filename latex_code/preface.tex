Le présent mémoire constitue une compilation et une synthèse des travaux de recherche entrepris pendant la période de mon stage de fin d'études s'étalant de mars à septembre 2023. Cette initiative s'inscrit dans le cadre de ma participation au programme de formation de niveau Master 2 intitulé "Mathematical Imaging and Spatial Pattern Analysis", dispensé par l'École Nationale Supérieure des Mines de Saint-Étienne (EMSE). Parallèlement, ce mémoire assume également le rôle de rapport final en vue de l'accomplissement des exigences du diplôme d'ingénieur généraliste décerné par l'École Nationale Supérieure des Mines d'Alès (EMA). \\

\noindent Le projet << Réseaux de neurones morphologiques >> dont fait l'objet ce stage est porté conjointement par l'École Pour l'Informatique et les Techniques Avancées (EPITA), à travers Monsieur Guillaume TOCHON au sein du Laboratoire de Recherche de l'EPITA (LRE), et par l'École Nationale Supérieure des Mines de Paris (Mines Paris - PSL), à travers Monsieur Jesús ANGULO au sein du Centre de Morphologie Mathématique (CMM). Depuis plusieurs années, ce projet s'est développé continuellement entre différentes mains et a fait l'objet de plusieurs publications dans des revues à commité de lecture avant d'aboutir à ce stage, dont l'objectif est l'exploration et l'amélioration des réseaux déjà développés, et s'achevant sur ce mémoire. \\

\noindent Plusieurs personnes ont contribué à la réalisation et au bon déroulement de ce stage. Je tiens tout d'abord à remercier Valentin PENAUD--POLGE, doctorant au CMM, pour m'avoir initialement partagé l'offre associée à ce projet, ainsi que pour les précieux conseils donnés avant le début du stage. Je souhaite ensuite remercier les professeurs et les doctorants du LRE pour leur soutien et leur collaboration tout le long de mon séjour au sein du laboratoire, en particulier Caroline MAZINI-RODRIGUES, Adam DESCARPENTRIES, Charles VILLARD, Michaël ROYNARD et Baptiste ESTEBAN pour leur aide quotidienne apportée et les agréables moments partagés. \\

\noindent J'aimerais également remercier Romain HERMARY, de qui j'hérite ce projet, pour la présentation de ses travaux et pour son aide continue apportée dans la compréhension du code. Je tiens ensuite à remercier Yann GAVET et Léo THÉODON de l'EMSE pour leur aide et leurs précieux conseils apportés durant mon séjour au LRE. Enfin, je souhaite remercier de tout cœur Guillaume TOCHON et Jesús ANGULO pour m'avoir supervisé et soutenu tout le long de ce stage, et sans qui ce projet n'existerait.

\vspace{1.2cm}
\noindent Antoine BOTTENMULLER

\vspace{0.4cm}
\makeatletter
\noindent Fait au Kremlin-Bicêtre (94270), le \@date 
\vfill
\makeatother