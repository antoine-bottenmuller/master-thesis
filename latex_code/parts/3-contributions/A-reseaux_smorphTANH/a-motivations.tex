%\vspace{1.4mm}
La littérature montre que les réseaux $\mathcal{S}$MorphNet, composés de couches $\mathcal{S}$Morph telles que décrites dans la partie état de l'art, sont les réseaux morphologiques existant les plus robustes dans leurs prédictions, à la fois en terme de perte globale \textit{loss} (précision des images prédites par rapport aux images cibles), en terme de \textit{RMSE} (similitude de la forme des noyaux \textit{w} des couches morphologiques avec celle de la fonction structurante cible), en terme d'exactitude de l'opération prédite ($\alpha << 0$ pour l'érosion, $\alpha >> 0$ pour la dilatation), et en terme de rapidité de convergence des réseaux lors de leur entraînement (avec le nombre d'époques nécessaire pour atteindre un minimum local dans la \textit{loss}, et donc pour atteindre la convergence du réseau). \\

\vspace{-1.5mm}
Cependant, bien que la couche $\mathcal{S}$Morph pallie de nombreuses contraintes liées à la formulation des couches $p$Conv et $\mathcal{L}$Morph (telles que le fait que, pour ces deux couches, $f$ et $w$ doivent être à valeurs positives, et nécessitent donc un rééchelonnage en entrée), il reste une principale contrainte dans ses formulations asympotiques. \\
%Cependant, bien que la couche $\mathcal{S}$Morph réussisse à résoudre de nombreuses contraintes liées aux couches $p$Conv et $\mathcal{L}$Morph (telles que le fait que $\mathcal{S}$Morph soit définie pour toute image $f$ et tout noyau $w$ à valeurs dans $\mathbb{R}$ tout entier, contrairement à $p$Conv et $\mathcal{L}$Morph pour lesquelles $f$ et $w$ doivent être à valeurs positives et nécessitent donc une mise en échelle positive), il reste une principale contrainte dans sa formulation asympotique (voir équation \ref{SMorph} et les formulations aux limites de $\alpha$). \\

\vspace{-2.0mm}
\noindent En effet, lorsque $\alpha \rightarrow +\infty$, on a pour $x \in I$ : $\lim_{\alpha \rightarrow +\infty} \mathcal{S}\text{Morph}(f,w,\alpha)(x) = \left ( f \oplus w \right )(x)$, donc $\mathcal{S}\text{Morph}(f,w,\alpha)$ tend bien vers la véritable image dilatée de $f$ par la fonction structurante $w$ quand $\alpha \rightarrow +\infty$. Mais, à l'inverse, lorsque $\alpha \rightarrow -\infty$, on a pour $x \in I$ : $\lim_{\alpha \rightarrow -\infty} \mathcal{S}\text{Morph}(f,w,\alpha)(x) = \left ( f \ominus -\breve{w} \right )(x)$, donc $\mathcal{S}\text{Morph}(f,w,\alpha)$ tend vers l'image érodée de $f$ par la fonction structurante $-\breve{w}$ quand $\alpha \rightarrow -\infty$, avec $-\breve{w}$ la \textit{symétrique centrale} et la \textit{négation} de $w$. Or, le but est d'avoir $\lim_{\alpha \rightarrow -\infty} \mathcal{S}\text{Morph}(f,w,\alpha)(x) = \left ( f \ominus w \right )(x)$. \\

\vspace{-1.4mm}
Une reformulation de $\mathcal{S}$Morph doit donc être pensée, afin d'abord de pallier ce problème de signe des poids $w$. Cette nouvelle formule ne doit cependant pas générer d'irrégularités ou de trop hautes variations par rapport à la formule de $\mathcal{S}$Morph, pour éviter l'apparition de minima locaux en plus dans la fonction de perte \textit{loss}, afin d'éviter le risque que les réseaux associés convergent moins bien (vers ces minima locaux-là). Il est donc requis que la nouvelle formule soit à minima davantage lisse (variations plus faibles) que $\mathcal{S}$Morph en fonction des variables $w$ et $\alpha$. \\

\vspace{-2.0mm}
\noindent Une nouvelle couche avec une nouvelle formulation satisfaisant les objectifs décrits a été ainsi pensée et initiée par Romain Hermary. Il s'agit de $\mathcal{S}$MorphTanh. La sous-partie suivante présente la formulation de cette nouvelle couche et ses propriétés. \\

\vspace{-1.3mm}
Il serait possible, sur la même idée, de pallier le problème de symétrie $\breve{w}$ avec une transition continue et douce de $w$ vers sa symétrie $\breve{w}$ quand $\alpha$ devient négatif au voisinage de $0$. Mais il s'agit là d'un travail futur. Nous nous concentrons d'abord sur la résolution de ce problème de signe devant $w$ avec cette nouvelle formule $\mathcal{S}$MorphTanh.
