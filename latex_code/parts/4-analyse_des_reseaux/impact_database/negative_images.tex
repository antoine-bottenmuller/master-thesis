On considère ici les images des banques MNIST et FashionMNIST en \textit{négatif}, c'est-à-dire qu'on leur applique : $f \mapsto 1-f$. Ainsi, les propriétés de chacune de ces deux banques d'images sont inversées : les images de MNIST en négatif comportent des objets (parties du support dont la valeur est proche de 1) très épais qui s'étalent le long des quatre bords du support et occupent une grande partie de l'espace sur ce dernier, englobant un fond (parties du support de valeur proche de 0) filamentaire fin et centré sur le support, qui disparaît facilement avec une dilatation ; tandis que les images de FashionMNIST en négatif comportent plutôt des objets qui s'étalent également le long des bords, mais qui, à l'inverse de MNIST, sont fins et disparaissent facilement avec une érosion, et qui entourent un fond épais centré sur le support. \\

\vspace{-1.6mm}
\noindent En relançant les mêmes entraînements que précédemment sur $\mathcal{S}$MorphNetTanh à deux couches morphologiques muni d'un partage de poids doux, mais avec cette fois-ci les deux jeux de données MNIST et FashionMNIST en négatif, on obtient les conclusions inverses par rapport à celles liées aux images en positif : sur MNIST, les résultats de convergence du réseau sont ici globalement meilleurs avec l'opération cible d'ouverture qu'avec la fermeture, et sur FashionMNIST, à l'inverse, les résultats sont globalement meilleurs pour la fermeture que pour l'ouverture (ce dernier cas présentant les mêmes défauts et irrégularités dans la forme des noyaux $w$ que pour les résultats de fermeture avec les images en positif). Ce qui est cohérent avec les caractéristiques inversées des images négatives par rapport aux images positives.
%(décrites ci-dessus) 

%%%

%-> Parler des expériences faites en négatif (là, c'est l'érosion qui est effectivement moins bonne que la dilatation ... l'efficacité entre les différentes opérations sont inversées, ce qui est cohérent, car les éléments sur les banques d'images sont inversés en négatifs. Si on peut faire la distinction entre un élément sur une image et le support (fond) de l'image, eh bien les images en positif avec de petits éléments deviennent, en négatif, des images avec de gros éléments et de petits trous.)
%Dire aussi qu'on obtient, avec les images d'entraînement en négatif, les résultats de performance inverses entre ouverture et fermeture (et entre érosion et dilatation)!!!